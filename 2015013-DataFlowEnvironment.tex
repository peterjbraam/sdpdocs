%rubber: module pdflatex
\documentclass[11pt,a4paper]{article}
\usepackage{microtype}\usepackage{mathptmx}
\usepackage{sdp_doc} % SDP style file
\usepackage[english]{babel} 
\usepackage{listings}
\usepackage[pdfborderstyle={/S/U/W 1}]{hyperref}

\usepackage{amsthm}
\usepackage{thmtools}
\usepackage[dvipsnames]{xcolor}
\usepackage{algpseudocode}

\declaretheoremstyle[spaceabove=6pt, spacebelow=6pt,
headfont=\normalfont\bfseries,
notefont=\mdseries, notebraces={(}{)},
bodyfont=\normalfont,
postheadspace=1em,
headpunct=\\,
notebraces=\ \ ,
qed=]{ExampleStyle}
\declaretheorem[style=ExampleStyle,shaded={rulecolor=Lavender,
rulewidth=1pt, margin=10pt, bgcolor={rgb}{0.98,0.98,1}}]{Example}




\definecolor{antiquewhite}{rgb}{0.98, 0.92, 0.86}

\lstset{ %
  backgroundcolor=\color{antiquewhite},   % choose the background color; you must add \usepackage{color} or \usepackage{xcolor}
  basicstyle=\footnotesize,        % the size of the fonts that are used for the code
  captionpos=b,                    % sets the caption-position to bottom
  frame=single}


%%%%%%%%% START OF USER SETTINGS %%%%%%%%%%%%%%%%%%

% Enter here some information needed to fill in the template Title to appear
% on the front pages (will be filled in via the \sdpfrontpage command)
\newcommand{\bigdoctitle}{SDP Dataflow Architecture Study\xspace}
% Title to go in the "Document Status Sheet" Document number
\newcommand{\docnr}{RP\_A0999\xspace}
% Context
\newcommand{\context}{(SDP Work Package)}
% Revision
\newcommand{\revision}{0.92\xspace}
% Author(s)
\newcommand{\docauthor}{P.\ Braam, S.\ Zefirov\xspace}
% Lead author (goes in the footer)
\newcommand{\leadauthor}{P.\ Braam\xspace}
% Release
\newcommand{\release}{1.0\xspace}
% Date of the document release, format: Month YYYY (e.g., August 2008)
\newcommand{\docudate}{2016-03-01\xspace}
% Document classification
\newcommand{\classification}{Unrestricted}
% Status of the document (draft/final/etc.)
\newcommand{\docstatus}{Draft\xspace}


% Table with signatures

\newcommand{\signaturetable}{
  \begin{tabularx}{\textwidth}{|X|X|X|}
      \hline
      Name & Designation & Affilitation\\
      \hline
      B. Nikolic \\
      \hline
      Signature \& Date: & & \\
      & & \\
      & & \\
      \hline
      Name & Designation & Affilitation\\
      \hline
      & & \\
      \hline
      Signature \& Date: & & \\
      & & \\
      & & \\
      \hline
  \end{tabularx}
}

  % Table with version numbers
  \newcommand{\versiontable}{
  \begin{tabularx}{\textwidth}{|X|X|X|X|}
        \hline
        \bf{Version} & {\bf Date of issue} & {\bf Prepared by} & {\bf Comments}\\
        \hline
        1.0 & & & \\
        \hline
      \end{tabularx}
  }

% Table with affiliations
\newcommand{\organisationtable}{
\begin{center}
 \sffamily{\bf ORGANISATION DETAILS}\end{center}
    \begin{table}[htbp]
      \centering
      \begin{tabular}[htbp]{|l|l|}
        \hline
        Name & Science Data Processor Consortium\\
        \hline
      \end{tabular}
    \end{table}
  }

%%%%%%%%%%%%% END OF USER SETTINGS %%%%%%%%%%%%%%%%%%%



\begin{document}

% load automatic pages
\sdpfrontpage

\sdptableofcontents

\sdplistoffigures

\sdplistoftables

% Add here the executive summary
\sdpsummary

This refines the SDP Summary Architecture Document by working out more detailed requirements and illustrating their implementation in the architecture and in some available data flow systems.  After giving definitions, the requirements are summarized in a table (external to this document), and some 20 of these requirements are studied in more detail through sample discussions of possible implementation (i) conceptually using the architectural concepts, (ii) in Regent / Legion (iii) in the DNA DSL.  We end the paper by mapping the requirements to some of the components in the SDP system.


(Section~\ref{sec:dataflow-definitions}) contains further definitions used in our description of data flow requirements and programs. 
(Section~\ref{sec:dataflow-requirements}) refers to the table and has little content.    We then present a significant set of sample data flow programs in 
(Section~\ref{sec:dataflow-examples}). The final section describes how our work is relevant to elements of the SDP dataflow, 
(Section~\ref{sec:dataflow-sdp-application}).


\sdpreferencedocs

\subsection*{Applicable Documents}

\iffalse
The following documents are applicable to the extent stated herein. In the
event of conflict between the contents of the applicable documents and this
document, \emph{the applicable documents} shall take precedence.

\begin{center}{
\begin{tabularx}{\textwidth}{|X|X|}
    \hline
    \bf{Reference} & \bf{Reference}\\
    \bf{Number} & \\
    \hline
    AD01 & Science Data Processor Architecture\\\hline
\end{tabularx}}
\end{center}

\subsection*{Reference Documents}

The following documents are referenced in this document. In the event of
conflict between the contents of the referenced documents and this document,
\emph{this document} shall take precedence.

\begin{center}{
\begin{tabularx}{\textwidth}{|X|X|}
    \hline
    \bf{Reference} & \bf{Reference}\\
    \bf{Number} & \\
    \hline
    RD01 & Science Data Processor Pipelines Element Design Report\\\hline
    RD02 & COMP Element Design Report\\\hline
  \end{tabularx}}
\end{center}

\fi

% The actual content goes here
\newpage
\section{Introduction}

\section{Dataflow Definitions}
\label{sec:dataflow-definitions}

\subsection{Concepts in dataflow programming}


\begin{Example}[A simple dataflow program]

  A simple example is a program that reads two vector and computes the
  sum product of them, i.e., at high level:
\begin{algorithmic}
  \State $a\gets \textrm{readFile "fileA"}$;  $b\gets \textrm{readFile "fileB"}$ ; $c\gets  \textrm{+/} \quad a\times b$
\end{algorithmic}  


\begin{lstlisting}
masterActor(input:fileName1, 
            input:fileName2, 
            input:clusterArchitecture) - schedules the work and starts it
     ch1, ch2 = open(param:fileName1, N), open(param:fileName2, N)
     fork(resource:computeNode, 
          actor:computeActor, 
          channel: ch1, channel: ch2,
          output:collectorProc,
          crash:ignore) -- get input upon fork
     fork(collectorNode, process:collectorActor, input:computeNodes, crash:fail)
     startDataFlow
     result = wait(collectorNode)

computeActor
     in1=fork(nodes:local, 
              actor:chFileRead, 
              input:ch1,
              input:offset)
     in2=fork(nodes:local, 
              actor:chFileRead, 
              input:ch2,
              input:offset)
    (A1, A2)=wait(in1, in2);
    res=dotProductLocal(A1, A2)
    join(output, res)  -- sends res message to the collector, notifies
                       -- parent of clean exit

collectorActor
    [partialDP ] = wait(input)
    res = sum([partialDP])
    join(res, parent) -- default case, join a parent and exit

\end{lstlisting}


\end{Example}


\subsection{Dataflow languages}


\begin{enumerate}
  \item


\item 
\end{enumerate}


\section{Dataflow Requirements}
\label{sec:dataflow-requirements}

\begin{table}
\begin{tabular}{|p{0.1\textwidth}|p{0.25\textwidth}|p{0.25\textwidth}|p{0.25\textwidth}|}
 \hline
& Parsec & Swift/T & Legion\\\hline
Static Scheduling &
Yes, to the level of the node &
No, some pinning of variables to control nodes  &
Yes, via partitions
\\\hline
Dynamic Scheduling &
Within a node only&
Yes, fully dynamic&
Yes, in the runtime system\\\hline
Data-dependent control flow &
No & Yes & Yes \\\hline
Actors & Kernels (typically from BLAS) & Kernels, complex procedures or external processes & Procedures \\\hline
Channels &
Message passing on interconnect &
Message passing on interconnect or files on a shared filesystem & Message passing \\\hline
\end{tabular}
\caption{Comparison of the features of Parsec, SWIFT/T and Legion dataflow languages}.
\label{tab:dataflowcomparision}
\end{table}

\section{The SKA SDP Dataflow Programs}
\label{sec:dataflow-examples}

\subsection{Granularity of the SDP program}

\section{Application of Data Flow to SDP}
\label{sec:dataflow-sdp-application}

\end{document}

%%% Local Variables: 
%%% mode: latex
%%% TeX-master: t
%%% End: 
